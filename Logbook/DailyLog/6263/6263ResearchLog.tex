\DailyTitle{6263 Log (Septemper 29, 2010)}

\DailySection{Goals}

\begin{enumerate}
\item Catch up with Maria
\item Get some rest
\end{enumerate}

\DailySection{Summary List}

\begin{enumerate}
\item Back from vacation.  Trying to catch up.
\item Copied exotica hotline code to CMSDetNoiseLine package.  Nothing modified.
\item Setup new logbook in latex.
\item Start reading exotica hotline code.
\end{enumerate}

\DailySection{Latex logbook}

The main goal is to have a logbook that is easily searchable and scalable.
Original handwritten logbooks have the advantage of sketching ideas, but
is not suitable for searching or write texts with a lot of revisions.
In the near future I might start using scanners to scan sketches as pictures
and include them in the latex logbook.  To make it scalable, each day
is to have its own tex segment which can be included in a tex file that
does the structuring.  The title and (sub-) sections are newly defined
commands that can be reassigned in the structuring tex file.

The structure of each day is as follows:

\begin{enumerate}
\item Daily goals.
\item Summary of things done.
\item For each non-trivial item, write something about it.
\item Meeting notes.
\item Anything else worth noting.
\item Reflection.  What was done and what could be done better.
\item Progress on studying, summary on paper reading.
\item Minimum goal for the next workday.
\end{enumerate}

Not all of them need to be filled in.

\DailySection{Reading exotica hotline code}

The code is in package
\texttt{UserCode/ExoticaHotLine/src/HotlineSkimCode/RecoSkim}.
In the final configuration file, each filter is a module, and
there are various paths assembling them together.  In the end
the events are kept using the \texttt{SelectEvents} field in
\texttt{PoolOutputModule}.

Even though it need not be the case, it appears that all the
filters are implemented together as a \texttt{EDFilter} named \texttt{RecoSkim}.
Different filters are the same module with different parameters.
For Hcal noise we definitely can implement multiple filter modules.

There are two modules in the hotline code directory.  One is the
aforementioned \texttt{RecoSkim} filter, which looks like basic cut-based
selections with cut values specified in the configuration file.
The other one is an analyzer \texttt{HotlineSummary}, and it appears to
be printing various summary values from edm collections.
The printout is long....this module is probably only for debugging purposes.


\DailySection{Reflection}

Need to think through the purpose of hcal noise hotline.
I want to be able to estimate noise rate (of various type)
for any given run from the hotline.  Also it will be good to
include some kind of correlation with beam luminosity and/or
triggers.

On latex logbook, need to think about possible types of extensions
and how to implement them.  In principle the current framework
should be enough.


\DailySection{Goals for next work day}

\begin{enumerate}
\item Skim through Hcal noise meetings
\item Skim through vecbos meetings
\item Catch up with progress on the candle note and make a list of items to do
\item Move the daily latex logbook to svn
\item Catch Maria
\end{enumerate}


