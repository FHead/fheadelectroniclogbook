\DailyTitle{6347 Log (November 7, 2010)}

\DailySection{Summary List}

\begin{enumerate}
\item Started peak hunting project
\end{enumerate}

\DailySection{Pion gun study}

Started simulation jobs using pion gun (500000, PT = 1-500 GeV/c, flat $\eta-\phi$).
They should finish within one or two days.
Brainstorming on things to look at:

\begin{enumerate}
\item What is the percentage that will start showering in ECAL?
\item Energy in Ecal and Hcal as a function of transverse momentum
\item Transverse shower shape in Hcal and Ecal
\item Is it at all possible to see the mass peak from track and calorimeters?
\item If there is a kink in the track, how is the data represented?
\item What is the chance of decaying into a muon in-flight?  Can we see the kink in the track?  Is it compatible with expectation?
\item Is it possible to use momentum conservation together with the assumption that neutrino is massless to see the pion mass peak?
\item Any visible difference between $\pi^+$ and $\pi^-$?
\item Charge mis-identification rate
\end{enumerate}

Also started batch jobs for electrons, photons, pi zeros, protons and neutrons.

Note: \texttt{CMSSW} version 3.7.0 patch4 used, with \texttt{cmsDriver} option:
\texttt{SingleMuPt1\_cfi.py -s GEN,SIM,DIGI,L1,DIGI2RAW,HLT:GRun --conditions START37\_V6::All --datatier GEN-SIM-RAW --eventcontent RAWSIM --beamspot Realistic7TeVCollision}




