\DailyTitle{6433 Log (December 15, 2010)}

\DailySection{Summary List}

\begin{enumerate}
\item Writing short bunch spacing section in the Hcal note
\item An attempt at errorbar on data
\end{enumerate}

\DailySection{One attempt at errorbar on data}

Since direct square root of count does not make sense, I tried to see if I can tell anything with it.
What I want to measure is the 95\% (or some other percentage) C.L. from the probability density function $P(\theta | x)$,
where $\theta$ is the theoretical parameter (real count), and $x$ is the count we see.
To get $P(\theta|x)$, we can borrow the Bayesian formation:

\begin{equation}
P(\theta | x) = \dfrac{P(x | \theta) P(\theta)}{P(x)}.\nonumber
\end{equation}

The different ingredients are as follows:

\begin{enumerate}
\item $P(x)$ is a constant in this calculation, and it doesn't interfere with the calculation of percentages.
\item $P(x | \theta)$ is the probability of observing $x$ given than the real parameter is $\theta$.  This can be modelled well by an Poissonian distribution.
\item $P(\theta)$ is the tricky one.  For now I am assuming that it is a Gaussian distribution around some theory value with an uncertainty.
The center of Gaussian could well be the value predicted from the fit for a certain bin.
\end{enumerate}

Fortunately $P(\theta | x)$ has only single peak and is monotonically decreasing away from the peak.
So a program is written (using slower, error-proof method) to extract the 90\% limit from the distribution.
For example if we observe 0 events, and the prediction is $10 \pm 3$, then the 95\% band is 0-6.70929.

One can also assume a flat prior.....in which case the error band when we see in case of non-observation is always 0-1









